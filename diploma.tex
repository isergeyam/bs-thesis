\documentclass{mipt-thesis-bs}
% Следующие две строки нужны только для biblatex. Для inline-библиографии их следует убрать.
\usepackage{mipt-thesis-biblatex}
\usepackage{hyperref}
\usepackage{minted}
\usepackage{float}
\usepackage{textgreek}
\usepackage{mathtools}
\usemintedstyle{tango}
\addbibresource{diploma.bib}

\newcommand*\numeral[1]{\underline{#1}}
\newcommand*\combinator[1]{\boldsymbol{\operatorname{\MakeUppercase{#1}}}}
\newcommand{\rr}{\rightarrow}
\newcommand{\CC}{\mathbb{C}}
\newcommand{\RR}{\mathbb{R}}
\newcommand{\NN}{\mathbb{N}}
\newcommand{\ZZ}{\mathbb{Z}}
\newcommand{\OO}{\mathcal{O}}


\title{Верификация асимптотической оценки временной сложности в задачах динамического программирования}
\author{Григорянц С.\,А.}
\supervisor{Дашков Е.\,В.}
%\referee{Петров Д.\,Е.}       % требуется только для mipt-thesis-ms
\groupnum{7910б}
\faculty{Физтех-школа прикладной математики и информатики}
\department{Кафедра дискретной математики}

\begin{document}

\newtheorem{lemma}{Лемма}[section]
\newtheorem{corollary}{Следствие}[section]
\newtheorem{proposition}{Утверждение}[section]
\newtheorem{remark}{Замечание}[section]
\newtheorem{fact}{Факт}[section]
\newtheorem{example}{Пример}[section]
\newtheorem{definition}{Определение}[section]

\newtheoremstyle{break}
{\topsep}{\topsep}%
{\itshape}{}%
{\bfseries}{}%
{\newline}{}%
\theoremstyle{break}
\newtheorem{theorem}{Теорема}[section]


\frontmatter
\titlecontents

\mainmatter


\chapter{Введение}
Cегодня, все больше отраслей компьютерных технологий нуждаются в формальной верификации. В первую очередь, в этот список входят программы,
связанные с транспортом, коммуникацией, медициной, компьютерной безопасностью, криптографией и банковским делом.
В этих критических сферах,
последствия отсутствия формальной верификации в некоторых проектах привели к очень плачевным последствиям \cite{horror}.
Фундаментальным отличием формальной верификации от классического тестирования ПО заключается в том, что, в то время как классическое
тестирование проверяет систему лишь на каком-то подмножестве возможных входов, формальная верификация ПО позволяет утвеждать о
корректности системы на всех возможных входах. Такого вида гарантии конечно же дают несравнимо большую уверенность в корректности
работы данного ПО, даже учитвая то, что остальные части системы, такие как компилятор, аппаратное обеспечение, и,
в конце концов, само ПО, используемое для верификации, могут дать сбои. Ибо мы таким образом убеждаемся в том, что сам код, который
является частью ПО, ошибок не содержит. Можно привести в пример много проектов, использующих формальную верификацию для довольно сложных
систем. Например, в верификации компиляторов -- проект Jinja \cite{KleinN-TCS,KleinN-ACM}, проект Verisoft
\cite{Strecker_compilerverification,Leinenbach}, а также проект CompCert \cite{Xavier,CompCert}.
В блокчейне -- проект Scilla \cite{sergey2018scilla}. Таким образом, мы можем убедиться в том, что формальная верификация ПО
и вправду очень востребованна.
\par
Но наряду с верификацией программ, также встает вопрос верификации асимптотики применяемого алгоритма. Действительно, довольно часто
мы хотим убедиться не только в том, что алгоритм корректен, но также и в том, что его асимптотика соответствует нашим ожиданиям.
На самом деле, баги, связанные с асимтотикой алгоритма, могут возникать довольно часто только для конкретных входов, что делает
классический подход тестирования неприемлемым для их отыскания. например, рассмотрим следующую реализацию бинарного поиска
(на языке Python).
\pagebreak
\begin{figure}[H]
    \caption{Проблемный бинарный поиск}
    \label{code:bsearch}
    \begin{minted}{python}
# Requires t to be a sorted array of integers.
# Returns k such that i <= k < j and t[k] = v
# or -1 if there is no such k.
def bsearch(t, v, i, j):
    if j <= i:
        return -1
    k = i + (j-i) // 2
    if v == t[k]:
        return k
    elif v < t[k]:
        return bsearch(t, v, i, k)
    return bsearch(t, v, i+1, j)
\end{minted}
\end{figure}
Проблема этого кода состоит в том, что при попадании в правую часть списка, вместо того, чтобы рассматривать интервал $[k+1;j)$, мы
рассматриваем интервал \\
$[i+1;j)$. Конечно, сам алгоритм корректно реализует бинарный поиск, но асимптотика при вводе, скажем,
последнего элемента массива, превращается из логарифмической в линейную. На этом примере хорошо видно, что даже формальной верификации
алгоритма и классического тестирования его на проверку асимптотики работы не всегда гарантирует нам корректность этой самой асимптотики.
В связи с этим, возникает потребность в том, чтобы иметь также возможность формально верифицировать не только корректность алгоритма, но
и его асимптотику.
\par
В данной работе мы покажем, как формальная верификация алгоритмов и их асимптотик может быть реализована с помощью Сепарационной Логики
с временными кредитами \cite{base_article}. С помощью этой теории, мы формально верифицируем алгоритм нахождения наибольшей общей подпоследовательности
в системе интерактивных доказательств Coq \cite{the_coq_development_team_2021_4501022}.

\chapter{Теоретические сведения}
В этой главе мы вкратце изложим принцип работы фреймворка, разработанного Шагеро и др. \cite{base_article}.
\section{Формализация $\mathcal{O}$-нотации}
Первым шагом в построении нужного формализма является формализация $\OO$-нотации. В классическом понимании, когда мы пишем $f = \OO(g)$, то это означает,
что $\exists c\ \exists n\ \forall x \geq n\  |f(x)| \leq c|g(x)|$, причем область значения и область определения обычно предполагаются подмножествами $\RR$.
Стоит сразу отметить, что обозначение $f = \OO(g)$ может запутать, т.к. на самом деле данное отношение не является отношением эквивалентности, а лишь
предпорядком. Поэтому, мы будем придерживаться обозначения $f \preccurlyeq g$, и говорить, что $g$ доминирует $f$.
Нас будут интересовать также и функции многих переменных, поэтому нам нужно рассмотреть обобщение этого понятия. Сразу понятно, что мы можем обобщить область
значений на произвольное нормированное векторное пространство, но в нашей работе нам достаточно будет рассмотреть $\ZZ$, т.к. мы будем оценивать асимптотику программ,
то есть количество шагов.
\section{Фильтры}
Для обобщения области значений, стоит перефразировать определение $\OO$-нотации на естественном языке: существует $c$, такое, что
для любого достаточно большого $x$, $f(x) \leq c|g(x)|$. В таком виде, когда мы видим фразу ''достаточно большой $x$'' или, если сказать по-другому,
''для почти любого $x$'', на ум сразу приходят фильтры множеств, т.к. фильтры как раз инкапсулируют понятие ''больших множеств'' то есть множеств, содержащих
в каком-то смысле ''почти все'' элементы. В результате, мы можем сформулировать следующее определение:
\begin{definition}
	Пусть $A$ -- множество, $\FF$ -- фильтр на нем, $V$ -- нормированное векторное пространство. Пусть $f, g: A \to V$. Тогда, будем говорить, что
	$f$ доминируется $g$($f \preccurlyeq g$), если $\{x: |f(x)| \leq c|g(x)|\} \in \FF$.
\end{definition}
Из этого определения понятно, что для задания предпорядка доминирования необходимо задать фильтр на соответствующем множестве. Мы будем обозначать
получившееся отношение как $\preccurlyeq_{\FF}$, и опускать $\FF$, когда он понятен из контекста. \\
Для реализации этого определения в Coq, нам нужно cформулировать его на языке теории типов. Пусть $A$ -- это тип. Тогда, т.к. $\FF$ в случае фильтра на
множестве $A$ является подмножеством $2^{A}$, то есть элементом $\FF \in 2^{2^{A}}$, то если $A$ -- это тип, то $\FF$ имеет тип $\PP(\PP(A))$, где
$\PP(A) = A \to \text{Prop}$. Переведем теперь определение фильтра на язык теории типов. Для начала запишем его на языке теории множеств.
\begin{definition}
	Пусть $A$ -- множество. Тогда $\FF \in 2^{2^{A}}$ -- фильтр на $A$, если выполняются следующие условия:
	\begin{enumerate}
		\item $\FF \neq \varnothing$
		\item $\varnothing \notin \FF$
		\item $\forall X,Y \in \FF,\ X \cap Y \in \FF$
		\item $\forall X \in \FF,\ \forall Y \supseteq X,\ Y \in \FF$
	\end{enumerate}
\end{definition}
Теперь переведем его на язык теории типов. Для начала введем обозначение. Пусть $A$ -- тип, $P: A \to Prop$ -- предикат, и $\UU: \PP(\PP(A)) = A \to (A \to Prop)$.
Тогда положим $\UU x.P = \UU (\Pi_{x: A} P x): Prop$ -- утверждение о том, что \\ $P$ выполняется на множестве из $\UU$. Теперь мы готовы сформулировать
определения фильтра в теории типов.
\begin{definition}
	Пусть $A$ -- тип. Тогда $\FF: \PP(\PP(A))$ -- фильтр на $A$, если
	\begin{enumerate}
		\item $\left(P_{1} \Rightarrow P_{2}\right) \Rightarrow \mathbb{U} x . P_{1} \Rightarrow \mathbb{U} x . P_{2}$
		\item $\mathbb{U} x . P_{1} \wedge \mathbb{U} x . P_{2} \Rightarrow \mathbb{U} x .\left(P_{1} \wedge P_{2}\right)$
		\item $\mathbb{U} x . \text { True }$
		\item $\mathbb{U} x . P \Rightarrow \exists x . P$
	\end{enumerate}
\end{definition}
\section{Примеры фильтров}
На любом ЧУМ $(A, \leq)$, в котором у любых двух элементов существует верхняя грань, можно задать фильтр, порожденный данным порядком.
\begin{definition}
	Пусть $(A, \leq)$ -- ЧУМ, в котором у любых двух элементов существует верхняя грань.
	Тогда $\FF = \{Q \subseteq A: \exists x_0 \forall x \geq x_0\ x \in Q\}$ -- фильтр на $A$.
\end{definition}
Заметим сразу, что стандартный порядок на $\ZZ, \NN, \RR$ порождает фильтр,
относительно которого отношение доминирования совпадает с классическим определением доминирования.
\par
Вспомним, что мы хотели построить обобщение отношения доминирования для того, чтобы перевести его на функции многих переменных. Это делается с помощью произведения
фильтров. Существует несколько определений произведения фильтров. Мы будем пользоваться следующим.
\begin{definition}
	Пусть $\FF_1, \FF_2$ -- фильтры на множествах $A_1, A_2$ соответственно. Тогда,
	$\FF = \FF_1 \times \FF_2 = \{Q \subseteq A_1 \times A_2: \exists Q_1 \in \FF_1, Q_2 \in \FF_2,\ Q_1 \times Q_2 \subseteq Q\}$ -- фильтр на $A_1 \times A_2$.
\end{definition}
С помощью этой конструкции, мы получаем фильтр, а следовательно и предпорядок доминирования на $\NN^k, \ZZ^k, \RR^k$ и т.д.
Например, если взять стандартный фильтр на ЧУМе $(A, \leq)$, и рассмотреть его квадрат как фильтр на $A^2$, то мы получим следующее отношение доминирования
на функциях $f, g: A^2 \to Z$:\\
$f \preccurlyeq g \iff \exists c \in \NN, a_0, b_0 \in A:\ \forall a \geq a_0, b \geq b_0\ |f(a, b)| \leq c|g(a, b)|$. \\
Именно этим определением доминирования на $\ZZ^2$ мы будем пользоваться при доказательстве нашего результата.
\section{Свойства отношения доминирования}
Для доказательства утверждений о доминировании функций, нам нужны будут свойства этого отношения.
В первую очередь, сформулируем утверждение о сохранении доминирования при суммировании функций.
\begin{lemma}\label{lemma:sum}
	Пусть $f, g, h: A \to Z$, $\FF$ -- фильтр на $A$. Тогда, $f \preccurlyeq h \land g \preccurlyeq h \implies (f + g) \preccurlyeq h$.
\end{lemma}
Следующая лемма позволяет нам сводить сумму к максимуму и наоборот.
\begin{lemma}\label{lemma:sum_max}
	Пусть $f, g: A \to Z$, $\FF$ -- фильтр на $A$. Тогда, если $\FF x. f(x) \geq 0 \land \FF x. g(x) \geq 0$, то
	$\max(f, g) \preccurlyeq (f + g) \land (f+g) \preccurlyeq \max(f, g)$.
\end{lemma}
Следующая лемма позволяет нам доказывать утверждения о доминировании произведения функций.
\begin{lemma}\label{lemma:prod}
	Пусть $f_1, g_1, f_2, g_2: A \to Z$, $\FF$ -- фильтр на $A$. Тогда,\\
	$f_1 \preccurlyeq g_1 \land f_2 \preccurlyeq g_2 \implies f_1f_2 \preccurlyeq g_1g_2$.
\end{lemma}
Следующая лемма позволяет доказывать утверждения о доминировании кумулятивных сумм, которые возникают в циклах.
\begin{lemma}\label{lemma:cumul}
	Пусть $f, g: A \times Z \to Z$, $\FF$ -- фильтр на $A$, $\UU$ -- стандартный фильтр на $\ZZ$, $i_0 \in \ZZ$. Тогда, при следующих условиях
	\begin{enumerate}
		\item $\FF x. \forall i \geq i_0\ f(a, i) \geq 0$
		\item $\FF x. \forall i \geq i_0\ g(a, i) \geq 0$
		\item $\forall a \in A, f_a(i) = f(a, i)$ -- не убывает на $[i_0;\infty]$
	\end{enumerate}
	выполняется, что если $f \preccurlyeq_{\FF \times \UU} g$, то $f^* \preccurlyeq_{\FF \times \UU} g^*$, где $f^*(a, n) = \sum_{i=i_{0}}^{n} f(a, i)$.
\end{lemma}
\section{Формализация спецификации}
Итак, в процессе рассуждений выше, нам стало понятно, что нужно для формальной записи спецификации программы:
\begin{itemize}
	\item Множество, на котором определена функция стоимости(обозначим его за $A$).
	\item порядок на $A$, который генерирует соответствующий фильтр.
	\item Функция доминирования стоимости.
	\item Само утверждение спецификации, которое говорит о том, что данная программа выполняется за выданное время и дает корректный выход.
\end{itemize}
Для этого набора в \cite{base_article} был введен специальный тип $\text{specO}$:

\begin{figure}[H]
	\caption{Определение specO}
	\label{code:specO}
	\begin{minted}{coq}
Record specO
       (A : filterType) (le : A -> A -> Prop)
       (spec : (A -> Z) -> Prop)
       (bound : A -> Z) :=
  SpecO {
      cost : A -> Z;
      spec : spec cost;
      cost_nonneg : forall x, 0 <= cost x;
      cost_monotonic : monotonic le Z.le cost;
      cost_dominated : dominated A cost bound
    }.
  \end{minted}
\end{figure}
В этих обозначениях, $A$ -- множество, на котором определена функция стоимости, $\text{le}$ -- порядок на $A$, $\text{spec}$ --
тройка в Сепарационной Логике, утверждающая о корректности программы, а также о том, что она выполняется за время $\text{cost}$,
$\text{bound}$ -- функция доминирования стоимости. Мы видим, что от функции стоимости также требуется неотрицательность и монотонность.
В \cite{base_article} показано, почему эти условия необходимы. Давайте рассмотрим это утверждение на конкретном примере.
Пусть $\text{length}$ -- функция, вычисляющая длину списка за $\mathcal{O}(n)$. Тогда, спецификация будет выглядеть следующим образом:

\begin{figure}[H]
	\caption{Спецификация length}
	\label{code:length_spec}
	\begin{minted}{coq}
Theorem length_spec:
	specO Z_filterType Z.le  (fun cost ->
	forall A (l:list A), triple (length l)
	PRE ($ (cost |l|))
	POST (fun y -> [ y = |l| ]))
	(fun n -> n)
  \end{minted}
\end{figure}

Написанное выше означает, что рассматривается $\ZZ$ со стандартным фильтром на нем, и функция доминирования стоимости -- $f(n) = n$. Это означает, что
для реальной функции стоимости $\text{cost}$ выполняется $\text{cost} = \mathcal{O}(n)$.

\chapter{Результаты}
В этой главе мы изложим полученный результат, состоящий в верификации алгоритма LCS.
\section{Реализация алгоритма}
Начнем изложение с представления реализации самого алгоритма. Ввиду использования фреймворка CFML \cite{base_article},
алгоритм реализован на языке OCaml \cite{ocaml}.

\begin{figure}[H]
  \caption{Реализация LCS}
  \label{code:lcs_impl}
  \begin{minted}{ocaml}
let lcs (a : int array) (b : int array) : int array =
  let n = Array.length a in
  let m = Array.length b in
  let c = Array.make ((n+1)*(m+1)) [] in
  for i = 1 to n do
    for j = 1 to m do
      if a.(i-1) = b.(j-1)
      then c.(i*(m+1) + j) <- List.append c.((i-1)*(m+1) + j - 1) [a.(i-1)]
      else if List.length c.((i-1)*(m+1) + j) > List.length c.(i*(m+1) + j - 1)
        then c.(i*(m+1) + j) <- c.((i-1)*(m+1) + j)
        else c.(i*(m+1) + j) <- c.(i*(m+1) + j - 1)
      done
    done; 
  Array.of_list c.((n+1)*(m+1)-1);;
\end{minted}
\end{figure}

Здесь приведена реализация классического алгоритма LCS, в котором с помощью динамического программирования вычисляется массив
$c[i][j]$, который содержит $LCS(A_i, B_j)$, где $A_i = a_1 \ldots a_i$, $B_j = b_1 \ldots b_j$.
Формула для вычисления $LCS(A_i, B_j)$ из предыдущих значений следующая:

$$
  L C S\left(A_{i}, B_{j}\right)=\left\{\begin{array}{ll}
    \emptyset                                                                              & \text { if } i=0 \text { or } j=0                    \\
    L C S\left(A_{i-1}, B_{j-1}\right) x_{i}                                               & \text { if } i, j>0 \text { and } a_{i}=b_{j}        \\
    \max \left\{L C S\left(A_{i}, B_{j-1}\right), L C S\left(A_{i-1}, B_{j}\right)\right\} & \text { if } i, j>0 \text { and } a_{i} \neq b_{j} .
  \end{array}\right.
$$

Но стоит отметить, что вместо двумерного массива $c$ используется одномерных массив размера $|a| \times |b|$. Это сделано из-за того,
что в оригинальной своей версии CFML \cite{base_article}, в языке пропозициональных высказываний о куче (тип \textbf{hprop})
не поддерживается квантора всеобщности, а только квантор существования (\textbf{\textbackslash HExists}). В случае двумерного массива
квантов всеобщности необходим для построения утвержений инварианте цикла.
Добавления квантора всеобщности в эту систему является не такой простой задачей, как может показаться на первый взгляд, поэтому
был использован одномерный массив.

\section{Формализация утверждений об Lcs}
Для того, чтобы сформулировать теорему об корректности работы и асимптотики алгоритма LCS, описанного выше, кроме инструментария
CFML, также необходима и некая формализация в Coq самого понятия LCS. Делается это с помощью так называемых индуктивных высказываний.
Определим сначала высказывание подпоследовательность, то есть $\text{SubSeq}\ l_1\ l_2$ означает, что $l_1$ является подпоследовательностью
$l_2$. Это утверждение формализуется следующим способом:


\begin{figure}[H]
  \caption{Определение SubSeq}
  \label{code:subseq_prop}
  \begin{minted}{coq}
Inductive SubSeq {A:Type} : list A -> list A -> Prop :=
 | SubNil (l:list A) : SubSeq nil l
 | SubCons1 (x:A) (l1 l2:list A) (H: SubSeq l1 l2) : SubSeq l1 (x::l2)
 | SubCons2 (x:A) (l1 l2:list A) (H: SubSeq l1 l2) : SubSeq (x::l1) (x::l2).
  \end{minted}
\end{figure}

Теперь мы также можем определить утверждение $\text{Lcs}\ l\ l_1\ l_2$, которое будет означать, что $l$ является наибольшей
общей подпоследовательностью $l_1$ и $l_2$:

\begin{figure}[H]
  \caption{Определение Lcs}
  \label{code:lcs_prop}
  \begin{minted}{coq}
Definition Lcs {A: Type} l l1 l2 :=
  SubSeq l l1 /\ SubSeq l l2 /\ 
  (forall l': list A, SubSeq l' l1 /\ SubSeq l' l2 -> length l' <= length l). 
  \end{minted}
\end{figure}

В этом утверждении просто проверяется, что во-первых $l$ и вправду является подпоследовательностью $l_1$ и $l_2$, а также то,
что любая другая их общая подпоследовательность $l'$ имеет длину, не большую $l$. Далее, для доказательства корректности работы
алгоритма, нужно будет убедиться в правильности построения $LCS(A_i, B_j)$. Вспомним правило построения:

$$
  L C S\left(A_{i}, B_{j}\right)=\left\{\begin{array}{ll}
    \emptyset                                                                              & \text { if } i=0 \text { or } j=0                    \\
    L C S\left(A_{i-1}, B_{j-1}\right) x_{i}                                               & \text { if } i, j>0 \text { and } a_{i}=b_{j}        \\
    \max \left\{L C S\left(A_{i}, B_{j-1}\right), L C S\left(A_{i-1}, B_{j}\right)\right\} & \text { if } i, j>0 \text { and } a_{i} \neq b_{j} .
  \end{array}\right.
$$

По большому счету, нам просто нужно доказать два утверждения об Lcs, в зависимости от равенства последних символов строк. Эти утверждения
формализуются следующим способом:
\begin{figure}[H]
  \caption{Необходимые свойства Lcs}
  \label{code:lcs_main_lemma}
  \begin{minted}{coq}
Lemma lcs_app_eq: forall (l1 l2 l: list int) (x: int),
  Lcs l l1 l2 -> Lcs (l & x) (l1 & x) (l2 & x). 

Lemma lcs_app_neq: forall (l1 l2 l l': list int) (x y : int),
  x <> y -> Lcs l (l1&x) l2 -> Lcs l' l1 (l2&y) -> length l' <= length l ->
  Lcs l (l1&x) (l2&y). 
  \end{minted}
\end{figure}
Для доказательства этих основных утверждений об Lcs, были использованы следующие вспомогательные леммы, устанавливающие свойства
SubSeq и Lcs:
\begin{figure}[H]
  \caption{Вспомогательные утверждения}
  \label{code:lcs_other_lemma}
  \begin{minted}{coq}
Lemma subseq_cons: forall A l1 l2 (x : A), SubSeq (x::l1) l2 -> SubSeq l1 l2. 

Lemma subseq_app: forall A l1 l2 (x : A), SubSeq l1 l2 -> SubSeq (l1 & x) (l2 & x). 

Lemma subseq_nil: forall A (l : list A), SubSeq l nil -> l = nil. 

Lemma subseq_length: forall (l a: list int), SubSeq l a -> length l <= length a. 

Lemma subseq_cons_l: forall A l1 l2 (x : A),  SubSeq (x :: l1) l2 <-> 
  exists l2' l2'', l2 = l2' & x ++ l2'' /\ SubSeq l1 l2''. 

Lemma subseq_last_head: forall l1 l2 (x y : int), 
  SubSeq (l1 & x) (l2 & y) -> SubSeq l1 l2. 

Lemma subseq_app_r: forall (l l1 l2: list int), SubSeq l l1 -> SubSeq l (l1 ++ l2). 

Lemma subseq_last_case: forall l l1 (x : int), SubSeq l (l1 & x) ->
  (SubSeq l l1) \/ (exists l', l = l' & x /\ (SubSeq l' l1)). 

Lemma subseq_last_case: forall l l1 (x : int), SubSeq l (l1 & x) ->
  (SubSeq l l1) \/ (exists l', l = l' & x /\ (SubSeq l' l1)). 

Lemma subseq_last_neq: forall l l1 l2 (x y : int), x <> y -> SubSeq l (l1 & x) -> 
  SubSeq l (l2 & y) -> (SubSeq l l1) \/ (SubSeq l l2). 

Lemma lcs_nil_nil: forall A (l: list A), Lcs nil nil l. 

Lemma lcs_symm: forall A (l l1 l2 : list A), Lcs l l1 l2 <-> Lcs l l2 l1. 
  \end{minted}
\end{figure}

\section{Формулировка основной теоремы}
Теперь мы практически готовы сформулировать основную теорему. Осталось только задать отношение на фильтре, т.к. для формулировки
нам нужен фильтр над $\ZZ^2$. Нам необходим стандартный фильтр на $\mathbb{Z}^2$, получающийся как произведение двух
стандартных фильров над $\ZZ$:

\begin{figure}[H]
  \caption{Определение фильтра}
  \label{code:filter_definition}
  \begin{minted}{coq}
Definition ZZle (p1 p2 : Z * Z) :=
  let (x1, y1) := p1 in
  let (x2, y2) := p2 in
  1 <= x1 <= x2 /\ 0 <= y1 <= y2.
  \end{minted}
\end{figure}

Теперь, можно сформулировать основную теорему:
\begin{figure}[H]
  \caption{Формулировка основной теоремы}
  \label{code:main_result_definition}
  \begin{minted}{coq}
Lemma lcs_spec:
  specO
    (product_filterType Z_filterType Z_filterType)
    ZZle
  ( fun cost =>
  forall (l1 l2 : list int) p1 p2,
  app lcs [p1 p2]
  PRE (\$(cost (LibListZ.length l1, LibListZ.length l2)) \*
  p1 ~> Array l1 \* p2 ~> Array l2)
  POST (fun p => Hexists (l : list int), p ~> Array l \* \[Lcs l l1 l2]))
  (fun '(n,m) => n * m).
  \end{minted}
\end{figure}

\chapter{Заключение}
В нашей работе мы на примере формальной верификации корректности алгоритма LCS и его асимптотики применили методы, описанные в \cite{base_article}
на практике, формально верифицировав утверждение \ref{code:main_result_definition}.
В процессе применения этой библиотеки, возникли некоторые проблемы с отсутствием квантора всеобщности для предиката кучи,
а именно, без него не получается строить утверждения о двумерных, и, следовательно, многомерных списках. Одним из вариантов развития
этой работы может послужить реализация предиката \textcolor{blue}{Hforall}, а также доказательство его различных свойств и интеграция в тактику
\textcolor{blue}{hsimpl}. Этот предикат был бы очень полезен не только при работе с многомерными массивами, но и во многих других случаях.
Еще одним интересным вариантом развития идей формальной верификации асимптотики алгоритмов является формальная верификация асимптотики
вероятностных алгоритмов. В то время как формализация проверки корректности вероятностных алгоритмов изучается довольно активно
(\cite{AUDEBAUD2009568}, \cite{tassarotti2017verifying}, \cite{10.1007/978-3-319-46750-4_5}), проверка асимптотики вероятностных алгоритмов
до сих пор практически не рассматривалась. Мы надеемся, что рассмотренные в этой работе методы могут быть обобщены и на вероятностные алгоритмы.


\chapter{Полная формализация результата}
\begin{minted}{coq}
Set Implicit Arguments.
Require Import TLC.LibTactics.
Require Import TLC.LibListZ.
(* Load the CFML library, with time credits. *)
Require Import CFML.CFLibCredits.
Require Pervasives_ml.
Require Array_ml.
Require Import Pervasives_proof.
Require Import ArrayCredits_proof.
(* Load the big-O library. *)
Require Import Dominated.
Require Import UltimatelyGreater.
Require Import Monotonic.
Require Import LibZExtra.
Require Import DominatedNary.
Require Import LimitNary.
Require Import Generic.
(* Load the custom CFML tactics with support for big-Os *)
Require Import CFMLBigO.
(* Load the CF definitions. *)
Require Import Lcs_flat_ml.

Open Scope liblist_scope.

Local Ltac auto_tilde ::= try solve [ auto with maths | false; math ].

Local Ltac my_invert H := inversion H; subst; clear H.

Lemma cons_injective: forall {A} x (l1 l2: list A), l1 = l2 -> x :: l1 = x :: l2. 
Proof.
  intros. rewrite H. reflexivity. 
Qed.


Lemma take_plus_one: forall (i : nat) (l: list int), 
  1 <= i <= length l -> take i l = take (i - 1) l & l[i-1]. 
Proof.
  intros. generalize dependent i. induction l. 
  - intros. rewrite length_nil in H. auto_false~. 
  - intros. rewrite take_cons_pos. destruct i. auto_false~. destruct i. 
    rewrite take_zero. rewrite take_zero. rewrite app_nil_l. 
    rewrite read_zero.  reflexivity. rewrite take_cons_pos. 
    rewrite read_cons_case. case_if. auto_false~. rewrite last_cons. 
    apply cons_injective. remember (S (S i) - 1) as i'. 
    remember (to_nat i') as i''. 
    assert (i' = i''). rewrite Heqi''. symmetry. apply to_nat_nonneg. math. 
    rewrite H0. apply IHl. rewrite <- H0. subst. rewrite length_cons in H. math. 
    math. math. 
Qed.

Lemma last_head: forall (l: list int), length l > 0 -> 
  exists l' x, l = l' & x. 
Proof.
  intros. exists (take (length l - 1) l) l[(length l) - 1]. 
  rewrite <- take_full_length at 1. apply take_plus_one. math. 
Qed.

Inductive SubSeq {A:Type} : list A -> list A -> Prop :=
 | SubNil (l:list A) : SubSeq nil l
 | SubCons1 (x:A) (l1 l2:list A) (H: SubSeq l1 l2) : SubSeq l1 (x::l2)
 | SubCons2 (x:A) (l1 l2:list A) (H: SubSeq l1 l2) : SubSeq (x::l1) (x::l2).

Lemma subseq_cons: forall A l1 l2 (x : A), SubSeq (x::l1) l2 -> SubSeq l1 l2. 
Proof.
  intros. remember (x::l1) as l1'. induction H. 
  - discriminate Heql1'. 
  - subst. constructor. apply IHSubSeq. reflexivity. 
  - my_invert Heql1'. constructor. assumption. 
Qed.

Lemma subseq_app: forall A l1 l2 (x : A), SubSeq l1 l2 -> SubSeq (l1 & x) (l2 & x). 
Proof.
  intros. induction H. 
  - induction l. 
    + rewrite last_nil. apply SubCons2. apply SubNil. 
    + rewrite last_cons. apply SubCons1. assumption. 
  - rewrite last_cons. apply SubCons1. assumption. 
  - rewrite last_cons. rewrite last_cons. apply SubCons2. assumption. 
Qed.

Lemma subseq_nil: forall A (l : list A), SubSeq l nil -> l = nil. 
Proof.
  intros. my_invert H. reflexivity. 
Qed.

Lemma subseq_length: forall (l a: list int), SubSeq l a -> length l <= length a. 
Proof.
  intros l. induction l. 
  - intros. rewrite length_nil. math. 
  - intros. my_invert H. 
    * apply subseq_cons in H0. apply IHl in H0. 
      rewrite length_cons. rewrite length_cons. math. 
    * apply IHl in H3. 
      rewrite length_cons. rewrite length_cons. math. 
Qed.

Lemma subseq_cons_l: forall A l1 l2 (x : A),  SubSeq (x :: l1) l2 <-> 
  exists l2' l2'', l2 = l2' & x ++ l2'' /\ SubSeq l1 l2''. 
Proof.
  split. generalize dependent x. generalize dependent l2. 
  {induction l1.  
  - intros l2. induction l2. 
    + intros. my_invert H. 
    + intros. my_invert H. 
      * apply IHl2 in H2. destruct H2 as [l2' [l2'' [H2 H3]]]. 
        exists (a :: l2') l2''. rewrite H2. auto. 
      * exists (@nil A) l2. auto. 
  - intros l2. induction l2. 
    + intros. my_invert H. 
    + intros. my_invert H. 
      * apply IHl2 in H2. destruct H2 as [l2' [l2'' [H2 H3]]]. 
        exists (a0 :: l2') l2''. rewrite H2. auto. 
      * exists (@nil A) l2. auto. 
  }
  {
    intros H. destruct H as [l2' [l2'' [H1 H2]]]. rewrite H1. generalize dependent l2. induction l2'. 
    - intros. rewrite last_nil. apply SubCons2. auto. 
    - intros. destruct l2. discriminate. 
      assert ((a :: l2') & x ++ l2'' = a :: (l2' & x ++ l2'')). reflexivity. 
      rewrite H in H1. injection H1 as H1. apply IHl2' in H0. rewrite H. 
      apply SubCons1. auto. 
  } 
Qed.


Lemma subseq_last_head: forall l1 l2 (x y : int), 
  SubSeq (l1 & x) (l2 & y) -> SubSeq l1 l2. 
Proof.
  intros l1. 
  induction l1. 
  - constructor. 
  - intros. rewrite last_cons in H. apply subseq_cons_l in H. 
    destruct H as [l2' [l2'' [H1 H2]]]. rewrite subseq_cons_l. 
    lets H3: subseq_length H2. rewrite length_last in H3. 
    assert (length l2'' > 0) by math. 
    lets H5: last_head l2'' H. destruct H5 as [l' [z H5]]. rewrite H5 in H1. 
    rewrite H5 in H2. apply IHl1 in H2. exists l2' l'. split. 
    assert (l2' & a ++ l' & z = (l2' & a ++ l') & z). rewrite last_app. 
    reflexivity. 
    rewrite H0 in H1. apply last_eq_last_inv in H1. destruct H1 as [H1 _]. 
    rewrite H1. 
    reflexivity. auto. 
Qed.

Lemma subseq_app_r: forall (l l1 l2: list int), SubSeq l l1 -> SubSeq l (l1 ++ l2). 
Proof.
  intros l. induction l. 
  - constructor. 
  - intros. rewrite subseq_cons_l in H. rewrite subseq_cons_l. 
    destruct H as [l2' [l2'' [H1 H2]]]. apply (IHl l2'' l2) in H2. 
    exists l2' (l2'' ++ l2). split. rewrite H1. rew_list. reflexivity. auto. 
Qed.

Lemma subseq_last_case: forall l l1 (x : int), SubSeq l (l1 & x) ->
  (SubSeq l l1) \/ (exists l', l = l' & x /\ (SubSeq l' l1)). 
Proof.
  intros l. induction l. 
  - left. constructor. 
  - intros. rewrite subseq_cons_l in H. 
    destruct H as [l2' [l2'' [H1 H2]]]. destruct l2''. 
    * apply subseq_nil in H2. rewrite app_nil_r in H1. subst. 
      apply last_eq_last_inv in H1. right. exists (@nil int). 
      destruct H1 as [H1 H2]. rewrite H2. split. rew_list. auto. constructor. 
    * remember (z :: l2'') as ll. assert (length ll > 0). 
      rewrite Heqll. rewrite length_cons. math. 
      lets M: (last_head ll H). destruct M as [l' [y H3]]. rewrite H3 in H1. 
      assert (l2' & a ++ l' & y = (l2' & a ++ l') & y). rewrite last_app. reflexivity. 
      rewrite H0 in H1. apply last_eq_last_inv in H1. destruct H1 as [H1 H4]. 
      rewrite H3 in H2. apply IHl in H2. destruct H2. 
      + left. rewrite subseq_cons_l. exists l2' l'. split; auto. 
      + destruct H2 as [l'0 [H2 H2']]. right. exists (a :: l'0). split. 
        rewrite last_cons. f_equal. rewrite H4. auto. rewrite subseq_cons_l. 
        exists l2' l'. split; auto. 
Qed.

Lemma subseq_last_neq: forall l l1 l2 (x y : int), x <> y -> SubSeq l (l1 & x) -> 
  SubSeq l (l2 & y) -> (SubSeq l l1) \/ (SubSeq l l2). 
Proof.
  intros. apply subseq_last_case in H0. destruct H0. 
  - left. auto. 
  - destruct H0 as [l' [H01 H02]]. apply subseq_last_case in H1. destruct H1. 
    + right. auto. 
    + destruct H0 as [l'' [H11 H12]]. rewrite H01 in H11. 
      apply last_eq_last_inv in H11. destruct H11 as [H1 H2]. auto_false. 
Qed.

Definition Lcs {A: Type} l l1 l2 :=
  SubSeq l l1 /\ SubSeq l l2 /\ 
  (forall l': list A, SubSeq l' l1 /\ SubSeq l' l2 -> length l' <= length l). 

Lemma lcs_nil_nil: forall A (l: list A), Lcs nil nil l. 
Proof.
  intros. unfold Lcs. split. constructor. split. constructor. intros. destruct H as [H _]. 
  apply subseq_nil in H. rewrite H. rewrite length_nil. math. 
Qed.

Lemma lcs_symm: forall A (l l1 l2 : list A), Lcs l l1 l2 <-> Lcs l l2 l1. 
Proof.
  intros. split. 
  - unfold Lcs. intros[H1 [H2 H3]]. split. auto. split. auto. 
    intros l' [H4 H5]. specialize (H3 l'). apply H3. split; auto.
  - unfold Lcs. intros[H1 [H2 H3]]. split. auto. split. auto. 
    intros l' [H4 H5]. specialize (H3 l'). apply H3. split; auto. 
Qed.

Lemma lcs_app_eq: forall (l1 l2 l: list int) (x: int),
  Lcs l l1 l2 -> Lcs (l & x) (l1 & x) (l2 & x). 
Proof.
  unfold Lcs. intros. destruct H as [H1 [H2 H3]]. split. 
  apply subseq_app. assumption. split. 
  apply subseq_app. assumption. 
  intros. destruct l'. rewrite length_nil. math. 
  remember (z :: l') as l''. 
  assert (HM: length l'' > 0). rewrite Heql''. rewrite length_cons. math. 
  lets M: last_head l'' HM. destruct M as [ll [y M]]. rewrite M. 
  rewrite length_last. rewrite length_last. destruct H as [Hl1 Hl2]. 
  rewrite M in Hl1. rewrite M in Hl2. 
  apply subseq_last_head in Hl1. apply subseq_last_head in Hl2. 
  assert (Hll: length ll <= length l) by auto. math. 
Qed. 

Lemma lcs_app_neq: forall (l1 l2 l l': list int) (x y : int),
  x <> y -> Lcs l (l1&x) l2 -> Lcs l' l1 (l2&y) -> length l' <= length l ->
  Lcs l (l1&x) (l2&y). 
Proof.
  unfold Lcs. intros. destruct H0 as [Hl_1 [Hl_2 Hl_3]]. 
  destruct H1 as [Hl'_1 [Hl'_2 Hl'_3]]. 
  split. auto. split. apply subseq_app_r. auto. 
  intros. destruct H0 as [H01 H02]. 
  assert (H' := H01). 
  eapply subseq_last_neq in H01. 
  3: {apply H02. } 2: {auto. } destruct H01. 
  - apply Hl_3. split; auto. 
  - assert (length l'0 <= length l' -> length l'0 <= length l) by math. 
    apply H1. apply Hl'_3. split; auto. 
Qed.



Definition ZZle (p1 p2 : Z * Z) :=
  let (x1, y1) := p1 in
  let (x2, y2) := p2 in
  1 <= x1 <= x2 /\ 0 <= y1 <= y2.

Lemma lcs_spec:
  specO
    (product_filterType Z_filterType Z_filterType)
    ZZle
  ( fun cost =>
  forall (l1 l2 : list int) p1 p2,
  app lcs [p1 p2]
  PRE (\$(cost (LibListZ.length l1, LibListZ.length l2)) \*
  p1 ~> Array l1 \* p2 ~> Array l2)
  POST (fun p => Hexists (l : list int), p ~> Array l \* \[Lcs l l1 l2]))
  (fun '(n,m) => n * m).
Proof.
  xspecO_refine straight_line. xcf. 
  xpay.  xapp~. intros. xapp~. intros. rewrite <- H. rewrite <- H0. xapp~. 
  assert (0 <= length l1) by (apply length_nonneg). 
  assert (0 <= length l2) by (apply length_nonneg). 
  rewrite <- H in H1. 
  rewrite <- H0 in H2. 
  { math_nia. }
  intros. weaken. 
  xfor_inv (fun (i:int) => 
    Hexists (x' : list (list int)),
    p1 ~> Array l1 \*
    p2 ~> Array l2 \*
    c ~> Array x' \*
    \[length x' = (n+1)*(m+1)] \*
    \[forall i1 i2 : int, 0 <= i1 < i -> 0 <= i2 <= m -> 
        Lcs x'[i1*(m+1) + i2] (take i1 l1 ) (take i2 l2) ] \* 
    \[forall i', i*(m+1) <= i' < (n+1)*(m+1) ->
        x'[i'] = nil ]
    ). 
  { math_nia. }
  2: {
    hsimpl.
    - intros. rewrite H1. rewrite read_make. reflexivity. apply~ int_index_prove. 
    - intros. assert (0 <= i1 < 1 -> i1 = 0) by math_nia. apply H4 in H2. 
      rewrite H2. rewrite take_zero. rewrite H1. rewrite read_make. 
      apply lcs_nil_nil. apply~ int_index_prove. math_nia. 
    - rewrite H1. apply length_make. math_nia. 
  }
  2: {
    intros. xapp~. apply~ int_index_prove. math_nia. 
    intros. xapp~. hsimpl_credits. specialize (H3 n m). 
    rewrite take_ge in H3. rewrite take_ge in H3. 
    assert (((n * (m + 1)%I)%I + m)%I = (((n + 1)%I * (m + 1)%I)%I - 1)%I) by math_nia. 
    rewrite H6 in H3. rewrite H5. apply H3. math_nia. math_nia. math_nia. math_nia. 
  }
  intros. xpay. xpull. intros. 
  xfor_inv (fun (j:int) => 
    Hexists (x' : list (list int)),
    p1 ~> Array l1 \*
    p2 ~> Array l2 \*
    c ~> Array x' \*
    \[length x' = (n+1)*(m+1)] \*
    \[forall i1 i2 : int, 0 <= i1 <= i -> 0 <= i2 <= m -> 
        i1*(m+1) + i2 < i*(m+1) + j -> 
        Lcs x'[i1*(m+1) + i2] (take i1 l1 ) (take i2 l2) ] \*
    \[forall i', i*(m+1) + j <= i' < (n+1)*(m+1) ->
        x'[i'] = nil ]
    ). 
  { math_nia. }
  2: {
    hsimpl. 
    - intros. apply H5. math_nia. 
    - intros. 
      remember (to_nat i1) as i1'. 
      remember (to_nat i) as i'. 
      assert (i1 = i1'). rewrite Heqi1'. symmetry. apply to_nat_nonneg. math. 
      assert (i = i'). rewrite Heqi'. symmetry. apply to_nat_nonneg. math. 
      assert ((i1' <= i')%nat) by math. 
      apply le_lt_eq_dec in H11. 
      destruct H11.
      + assert (i1 < i) by math. clear Heqi1' Heqi' l H9 H10 i' i1'. 
        apply H4; math. 
      + assert (i1 = i) by math. clear Heqi1' Heqi' e H9 H10 i' i1'. 
        rewrite lcs_symm. assert (x0[((i1 * (m + 1)%I)%I + 0)%I] = nil). 
        apply H5. math_nia. asserts_rewrite (i2 = 0). math_nia. 
        rewrite H9. apply lcs_nil_nil. 
    - assumption. 
  }
  2: {
    hsimpl. 
    - intros. apply H9. math_nia. 
    - intros. apply H8; math_nia. 
    - assumption. 
  }
  intros. xpay. xpull. intros. 
  xapp~. { apply~ int_index_prove. }
  intros. xapp~. { apply~ int_index_prove. }
  intros. xret. intros. xif. 
  {
    xapp~. { apply~ int_index_prove. }
    intros. xapp~. 
    { apply~ int_index_prove. math_nia. math_nia. } 
    intros. xret. intros. xapp~. 
    { apply~ int_index_prove. math_nia. math_nia. } 
    xpull. hsimpl_credits. 
    - intros. 
      remember (((i * (m + 1)) + i0)) as j. 
      remember x11__ as v. 
      assert ((x1[j:=v])[i'] = x1[i']). 
      (* TODO: WHY THE HECK read_update_neq does not work??? *)
      rewrite read_update_case. case_if; auto_false~. 
      apply~ int_index_prove. math_nia. 
      rewrite H16. apply H9. math_nia. 
    - intros. 
      remember (to_nat i1) as i1'. 
      remember (to_nat i) as i'. 
      assert (i1 = i1'). rewrite Heqi1'. symmetry. apply to_nat_nonneg. math. 
      assert (i = i'). rewrite Heqi'. symmetry. apply to_nat_nonneg. math. 
      assert ((i1' <= i')%nat) by math. 
      apply le_lt_eq_dec in H20. 
      destruct H20. 
      + assert (i1 < i) by math. 
        rewrite read_update_case. case_if. 
        assert (((i * (m + 1)%I)%I + i0)%I > ((i1 * (m + 1)%I)%I + i2)%I) by math_nia. 
        auto_false~. apply H8; math_nia. apply int_index_prove; math_nia. 
      + remember (to_nat i2) as i2'. 
        remember (to_nat (i0 + 1)) as i0'. 
        assert (i2 = i2'). rewrite Heqi2'. symmetry. apply to_nat_nonneg. math. 
        assert (i0 + 1 = i0'). rewrite Heqi0'. symmetry. apply to_nat_nonneg. math. 
        assert ((i2' < i0')%nat) by math_nia. 
        apply le_lt_eq_dec in H22. 
        destruct H22. 
        * assert (i2 < i0) by math. 
          rewrite read_update_case. case_if. 
          assert (((i * (m + 1)%I)%I + i0)%I > ((i1 * (m + 1)%I)%I + i2)%I) by math_nia. 
          auto_false~. apply H8; math_nia. apply int_index_prove; math_nia. 
        * assert (i1 = i) by math. assert (i2 = i0) by math. 
          rewrite read_update_case. case_if. rewrite H14. 
          rewrite H22. rewrite H19. rewrite take_plus_one. 
          rewrite <- H19. rewrite H20. rewrite take_plus_one. rewrite <- H20. 
          rewrite H23. rewrite H12. 
          rewrite <- H11. rewrite C. rewrite H10. rewrite <- H10. 
          apply lcs_app_eq. 
          rewrite H13. 
          asserts_rewrite (((((i - 1)%I * (m + 1)%I)%I + i0)%I - 1)%I = (((i - 1)%I * (m + 1)%I)%I + (i0%I - 1)%I)). 
          math. apply H8; math_nia. math_nia. math_nia. apply~ int_index_prove; math_nia. 
    - rewrite <- H7. apply length_update. 
  }
  {
    xapp~. { apply~ int_index_prove. math_nia. math_nia. }
    intros. xret. intros. xapp~. 
    { apply~ int_index_prove. math_nia. math_nia. } 
    intros. xret. intros. xif. 
    {
      xapp~. 
      { apply~ int_index_prove. math_nia. math_nia. } 
      intros. xapp~. 
      { apply~ int_index_prove. math_nia. math_nia. } 
      hsimpl_credits.
      {
        intros. 
        rewrite read_update_case. case_if; auto_false~. 
        apply int_index_prove; math_nia. 
      }
      - intros. 
        remember (to_nat i1) as i1'. 
        remember (to_nat i) as i'. 
        assert (i1 = i1'). rewrite Heqi1'. symmetry. apply to_nat_nonneg. math. 
        assert (i = i'). rewrite Heqi'. symmetry. apply to_nat_nonneg. math. 
        assert ((i1' <= i')%nat) by math. 
        apply le_lt_eq_dec in H22. 
        destruct H22. 
        + assert (i1 < i) by math. 
          rewrite read_update_case. case_if. 
          assert (((i * (m + 1)%I)%I + i0)%I > ((i1 * (m + 1)%I)%I + i2)%I) by math_nia. 
          auto_false~. apply H8; math_nia. apply int_index_prove; math_nia. 
        + remember (to_nat i2) as i2'. 
          remember (to_nat (i0 + 1)) as i0'. 
          assert (i2 = i2'). rewrite Heqi2'. symmetry. apply to_nat_nonneg. math. 
          assert (i0 + 1 = i0'). rewrite Heqi0'. symmetry. apply to_nat_nonneg. math. 
          assert ((i2' < i0')%nat) by math_nia. 
          apply le_lt_eq_dec in H24. 
          destruct H24. 
          * assert (i2 < i0) by math. 
            rewrite read_update_case. case_if. 
            assert (((i * (m + 1)%I)%I + i0)%I > ((i1 * (m + 1)%I)%I + i2)%I) by math_nia. 
            auto_false~. apply H8; math_nia. apply int_index_prove; math_nia. 
          * assert (i1 = i) by math. assert (i2 = i0) by math. 
            rewrite read_update_case. case_if. rewrite H16. 
            rewrite H20. rewrite take_plus_one. rewrite <- H20. 
            rewrite H22. rewrite take_plus_one. rewrite <- H22. 
            rewrite H24. rewrite H25. 
            rewrite <- H10. rewrite <- H11. 
            (* rewrite <- H11. rewrite C. rewrite H10. rewrite <- H10.  *)
            rewrite lcs_symm. 
            eapply lcs_app_neq. auto. rewrite H10. 
            rewrite <- H25. rewrite H22. rewrite <- take_plus_one. rewrite lcs_symm. 
            apply H8; math_nia. math_nia. rewrite lcs_symm. rewrite H11. rewrite H21. 
            rewrite <- take_plus_one. rewrite <- H21. apply H8; math_nia. math_nia. 
            asserts_rewrite (((i * (m + 1)%I)%I + (i0%I - 1)%I) = (((i * (m + 1)%I)%I + i0)%I - 1)%I ). math_nia. 
            rewrite <- H12. rewrite <- H14. rewrite <- H13. rewrite <- H15. math. math. math. 
            apply int_index_prove; math_nia. 
      - rewrite <- H7. apply length_update. 
    }
    {
      xapp~. 
      { apply~ int_index_prove. math_nia. math_nia. } 
      intros. xapp~. 
      { apply~ int_index_prove. math_nia. math_nia. } 
      hsimpl_credits. 
      {
        intros. 
        rewrite read_update_case. case_if; auto_false~. 
        apply int_index_prove; math_nia. 
      }
      - intros. 
        remember (to_nat i1) as i1'. 
        remember (to_nat i) as i'. 
        assert (i1 = i1'). rewrite Heqi1'. symmetry. apply to_nat_nonneg. math. 
        assert (i = i'). rewrite Heqi'. symmetry. apply to_nat_nonneg. math. 
        assert ((i1' <= i')%nat) by math. 
        apply le_lt_eq_dec in H22. 
        destruct H22. 
        + assert (i1 < i) by math. 
          rewrite read_update_case. case_if. 
          assert (((i * (m + 1)%I)%I + i0)%I > ((i1 * (m + 1)%I)%I + i2)%I) by math_nia. 
          auto_false~. apply H8; math_nia. apply int_index_prove; math_nia. 
        + remember (to_nat i2) as i2'. 
          remember (to_nat (i0 + 1)) as i0'. 
          assert (i2 = i2'). rewrite Heqi2'. symmetry. apply to_nat_nonneg. math. 
          assert (i0 + 1 = i0'). rewrite Heqi0'. symmetry. apply to_nat_nonneg. math. 
          assert ((i2' < i0')%nat) by math_nia. 
          apply le_lt_eq_dec in H24. 
          destruct H24. 
          * assert (i2 < i0) by math. 
            rewrite read_update_case. case_if. 
            assert (((i * (m + 1)%I)%I + i0)%I > ((i1 * (m + 1)%I)%I + i2)%I) by math_nia. 
            auto_false~. apply H8; math_nia. apply int_index_prove; math_nia. 
          * assert (i1 = i) by math. assert (i2 = i0) by math. 
            rewrite read_update_case. case_if. rewrite H16. 
            rewrite H20. rewrite take_plus_one. rewrite <- H20. 
            rewrite H22. rewrite take_plus_one. rewrite <- H22. 
            rewrite H24. rewrite H25. 
            rewrite <- H10. rewrite <- H11. 
            (* rewrite <- H11. rewrite C. rewrite H10. rewrite <- H10.  *)
            eapply lcs_app_neq. auto. rewrite H11. rewrite H21. rewrite <- take_plus_one. rewrite <- H21. 
            asserts_rewrite ((((i * (m + 1)%I)%I + i0)%I - 1)%I = ((i * (m + 1)%I)%I + (i0%I - 1)%I)). math. 
            apply H8; math.  math. 
            rewrite H10. 
            rewrite <- H25. rewrite H22. rewrite <- take_plus_one. rewrite <- H22. 
            apply H8; math_nia. math. rewrite <- H12. rewrite <- H14. rewrite <- H13. rewrite <- H15. math. math. math. 
            apply int_index_prove; math_nia. 
      - rewrite <- H7. apply length_update. 
    }
  }
  reflexivity. 
  cleanup_cost. 
  { equates 1; swap 1 2.
    { instantiate (1 := (fun '(x, y) => _)). apply fun_ext_1. intros [x y].
      rewrite !cumul_const'. rew_cost. reflexivity. }
    intros [x1 y1] [x2 y2] [H1 H2]. math_nia. }
  apply_nary dominated_sum_distr_nary; swap 1 2.
  dominated. 
  apply_nary dominated_sum_distr_nary.
  apply_nary dominated_sum_distr_nary.
  apply_nary dominated_sum_distr_nary.
  dominated. 
  { apply dominated_transitive with (fun '(x, y) => x * 1).
    - (* TODO: improve using some setoid rewrite instances? *)
      apply dominated_eq. intros [? ?]. math.
    - apply_nary dominated_mul_nary; dominated. 
  }
  { apply dominated_transitive with (fun '(x, y) => 1 * y).
    - (* TODO: improve using some setoid rewrite instances? *)
      apply dominated_eq. intros [? ?]. math.
    - apply_nary dominated_mul_nary; dominated. 
  }

  { eapply dominated_transitive.
    apply dominated_product_swap.
    apply Product.dominated_big_sum_bound_with.
    { apply filter_universe_alt. intros. rewrite~ <-cumul_nonneg. math_lia. }
    { monotonic. }
    { limit.  }
    simpl. dominated.

    now repeat apply_nary dominated_sum_distr_nary; dominated.
    repeat apply_nary dominated_sum_distr_nary; dominated.
    etransitivity. apply Product.dominated_big_sum_bound_with. 
    intros. apply filter_universe_alt. math_lia. 
    monotonic. limit. dominated. apply_nary dominated_sum_distr_nary; dominated. } 
Qed.

  \end{minted}


\backmatter

\printbib
% Следующие строки необходимо раскомментировать, а предыдущую закомментировать, если используется inline-библиография.
%\begin{thebibliography}{99}
%    \bibitem{langmuir26}
%        H. Mott-Smith, I. Langmuir. ``The theory of collectors in gaseous discharges''. \emph{Phys. Rev.} \textbf{28} (1926)
%\end{thebibliography}

% \chapter{Благодарности}

% Благодарности идут тут.

\end{document}