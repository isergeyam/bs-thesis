В этой главе мы изложим полученный результат, состоящий в верификации алгоритма LCS.
\section{Реализация алгоритма}
Начнем изложение с представления реализации самого алгоритма. Ввиду использования фреймворка CFML \cite{base_article},
алгоритм реализован на языке OCaml \cite{ocaml}.

\begin{figure}[H]
  \caption{Реализация LCS}
  \label{code:lcs_impl}
  \begin{minted}{ocaml}
let lcs (a : int array) (b : int array) : int array =
  let n = Array.length a in
  let m = Array.length b in
  let c = Array.make ((n+1)*(m+1)) [] in
  for i = 1 to n do
    for j = 1 to m do
      if a.(i-1) = b.(j-1)
      then c.(i*(m+1) + j) <- List.append c.((i-1)*(m+1) + j - 1) [a.(i-1)]
      else if List.length c.((i-1)*(m+1) + j) > List.length c.(i*(m+1) + j - 1)
        then c.(i*(m+1) + j) <- c.((i-1)*(m+1) + j)
        else c.(i*(m+1) + j) <- c.(i*(m+1) + j - 1)
      done
    done; 
  Array.of_list c.((n+1)*(m+1)-1);;
\end{minted}
\end{figure}

Здесь приведена реализация классического алгоритма LCS, в котором с помощью динамического программирования вычисляется массив
$c[i][j]$, который содержит $LCS(A_i, B_j)$, где $A_i = a_1 \ldots a_i$, $B_j = b_1 \ldots b_j$.
Формула для вычисления $LCS(A_i, B_j)$ из предыдущих значений следующая:

$$
  L C S\left(A_{i}, B_{j}\right)=\left\{\begin{array}{ll}
    \emptyset                                                                              & \text { if } i=0 \text { or } j=0                    \\
    L C S\left(A_{i-1}, B_{j-1}\right) x_{i}                                               & \text { if } i, j>0 \text { and } a_{i}=b_{j}        \\
    \max \left\{L C S\left(A_{i}, B_{j-1}\right), L C S\left(A_{i-1}, B_{j}\right)\right\} & \text { if } i, j>0 \text { and } a_{i} \neq b_{j} .
  \end{array}\right.
$$

Но стоит отметить, что вместо двумерного массива $c$ используется одномерных массив размера $|a| \times |b|$. Это сделано из-за того,
что в оригинальной своей версии CFML \cite{base_article}, в языке пропозициональных высказываний о куче (тип \textbf{hprop})
не поддерживается квантора всеобщности, а только квантор существования (\textbf{\textbackslash HExists}). В случае двумерного массива
квантов всеобщности необходим для построения утвержений инварианте цикла.
Добавления квантора всеобщности в эту систему является не такой простой задачей, как может показаться на первый взгляд, поэтому
был использован одномерный массив.

\section{Формализация утверждений об Lcs}
Для того, чтобы сформулировать теорему об корректности работы и асимптотики алгоритма LCS, описанного выше, кроме инструментария
CFML, также необходима и некая формализация в Coq самого понятия LCS. Делается это с помощью так называемых индуктивных высказываний.
Определим сначала высказывание подпоследовательность, то есть $\text{SubSeq}\ l_1\ l_2$ означает, что $l_1$ является подпоследовательностью
$l_2$. Это утверждение формализуется следующим способом:


\begin{figure}[H]
  \caption{Определение SubSeq}
  \label{code:subseq_prop}
  \begin{minted}{coq}
Inductive SubSeq {A:Type} : list A -> list A -> Prop :=
 | SubNil (l:list A) : SubSeq nil l
 | SubCons1 (x:A) (l1 l2:list A) (H: SubSeq l1 l2) : SubSeq l1 (x::l2)
 | SubCons2 (x:A) (l1 l2:list A) (H: SubSeq l1 l2) : SubSeq (x::l1) (x::l2).
  \end{minted}
\end{figure}

Теперь мы также можем определить утверждение $\text{Lcs}\ l\ l_1\ l_2$, которое будет означать, что $l$ является наибольшей
общей подпоследовательностью $l_1$ и $l_2$:

\begin{figure}[H]
  \caption{Определение Lcs}
  \label{code:lcs_prop}
  \begin{minted}{coq}
Definition Lcs {A: Type} l l1 l2 :=
  SubSeq l l1 /\ SubSeq l l2 /\ 
  (forall l': list A, SubSeq l' l1 /\ SubSeq l' l2 -> length l' <= length l). 
  \end{minted}
\end{figure}

В этом утверждении просто проверяется, что во-первых $l$ и вправду является подпоследовательностью $l_1$ и $l_2$, а также то,
что любая другая их общая подпоследовательность $l'$ имеет длину, не большую $l$. Далее, для доказательства корректности работы
алгоритма, нужно будет убедиться в правильности построения $LCS(A_i, B_j)$. Вспомним правило построения:

$$
  L C S\left(A_{i}, B_{j}\right)=\left\{\begin{array}{ll}
    \emptyset                                                                              & \text { if } i=0 \text { or } j=0                    \\
    L C S\left(A_{i-1}, B_{j-1}\right) x_{i}                                               & \text { if } i, j>0 \text { and } a_{i}=b_{j}        \\
    \max \left\{L C S\left(A_{i}, B_{j-1}\right), L C S\left(A_{i-1}, B_{j}\right)\right\} & \text { if } i, j>0 \text { and } a_{i} \neq b_{j} .
  \end{array}\right.
$$

По большому счету, нам просто нужно доказать два утверждения об Lcs, в зависимости от равенства последних символов строк. Эти утверждения
формализуются следующим способом:
\begin{figure}[H]
  \caption{Необходимые свойства Lcs}
  \label{code:lcs_main_lemma}
  \begin{minted}{coq}
Lemma lcs_app_eq: forall (l1 l2 l: list int) (x: int),
  Lcs l l1 l2 -> Lcs (l & x) (l1 & x) (l2 & x). 

Lemma lcs_app_neq: forall (l1 l2 l l': list int) (x y : int),
  x <> y -> Lcs l (l1&x) l2 -> Lcs l' l1 (l2&y) -> length l' <= length l ->
  Lcs l (l1&x) (l2&y). 
  \end{minted}
\end{figure}
Для доказательства этих основных утверждений об Lcs, были использованы следующие вспомогательные леммы, устанавливающие свойства
SubSeq и Lcs:
\begin{figure}[H]
  \caption{Вспомогательные утверждения}
  \label{code:lcs_other_lemma}
  \begin{minted}{coq}
Lemma subseq_cons: forall A l1 l2 (x : A), SubSeq (x::l1) l2 -> SubSeq l1 l2. 

Lemma subseq_app: forall A l1 l2 (x : A), SubSeq l1 l2 -> SubSeq (l1 & x) (l2 & x). 

Lemma subseq_nil: forall A (l : list A), SubSeq l nil -> l = nil. 

Lemma subseq_length: forall (l a: list int), SubSeq l a -> length l <= length a. 

Lemma subseq_cons_l: forall A l1 l2 (x : A),  SubSeq (x :: l1) l2 <-> 
  exists l2' l2'', l2 = l2' & x ++ l2'' /\ SubSeq l1 l2''. 

Lemma subseq_last_head: forall l1 l2 (x y : int), 
  SubSeq (l1 & x) (l2 & y) -> SubSeq l1 l2. 

Lemma subseq_app_r: forall (l l1 l2: list int), SubSeq l l1 -> SubSeq l (l1 ++ l2). 

Lemma subseq_last_case: forall l l1 (x : int), SubSeq l (l1 & x) ->
  (SubSeq l l1) \/ (exists l', l = l' & x /\ (SubSeq l' l1)). 

Lemma subseq_last_case: forall l l1 (x : int), SubSeq l (l1 & x) ->
  (SubSeq l l1) \/ (exists l', l = l' & x /\ (SubSeq l' l1)). 

Lemma subseq_last_neq: forall l l1 l2 (x y : int), x <> y -> SubSeq l (l1 & x) -> 
  SubSeq l (l2 & y) -> (SubSeq l l1) \/ (SubSeq l l2). 

Lemma lcs_nil_nil: forall A (l: list A), Lcs nil nil l. 

Lemma lcs_symm: forall A (l l1 l2 : list A), Lcs l l1 l2 <-> Lcs l l2 l1. 
  \end{minted}
\end{figure}

\section{Формулировка основной теоремы}
Теперь мы практически готовы сформулировать основную теорему. Осталось только задать отношение на фильтре, т.к. для формулировки
нам нужен фильтр над $\ZZ^2$. Нам необходим стандартный фильтр на $\mathbb{Z}^2$, получающийся как произведение двух
стандартных фильров над $\ZZ$:

\begin{figure}[H]
  \caption{Определение фильтра}
  \label{code:filter_definition}
  \begin{minted}{coq}
Definition ZZle (p1 p2 : Z * Z) :=
  let (x1, y1) := p1 in
  let (x2, y2) := p2 in
  1 <= x1 <= x2 /\ 0 <= y1 <= y2.
  \end{minted}
\end{figure}

Теперь, можно сформулировать основную теорему:
\begin{figure}[H]
  \caption{Формулировка основной теоремы}
  \label{code:main_result_definition}
  \begin{minted}{coq}
Lemma lcs_spec:
  specO
    (product_filterType Z_filterType Z_filterType)
    ZZle
  ( fun cost =>
  forall (l1 l2 : list int) p1 p2,
  app lcs [p1 p2]
  PRE (\$(cost (LibListZ.length l1, LibListZ.length l2)) \*
  p1 ~> Array l1 \* p2 ~> Array l2)
  POST (fun p => Hexists (l : list int), p ~> Array l \* \[Lcs l l1 l2]))
  (fun '(n,m) => n * m).
  \end{minted}
\end{figure}


\section{Доказательство основной теоремы}
Перейдем к изложению доказательства основной теоремы. Начнем с изложения основных идей при доказательстве алгоритмической корректности
алгоритма
\subsection{Доказательство алгоритмической корректности}
Ясно, что ключ к ее доказательству лежит в правильном выборе инвариантов
двух основных циклов алгоритма. Как было изложено ранее, эти циклы отвечают за вычисление $c[i][j] = LCS(A_i, B_j)$, то есть
инвариантами цикла является утверждение о том, что во всех предыдущих ячейках $c[i][j]$ записаны верные значения $LCS(A_i, B_j)$.
Может показаться, что этого достаточно, но на самом деле для формальной верификации нужно еще кое-что. Мы ведь не учли, что наш инвариант
должен оставаться верным и при первой итерации цикла, когда в $c[i][j]$ записан пустой список по умолчанию
(что верно, т.к. мы берем пустой префикс у списка). Поэтому, нам также необходимо добавить в инвариант утверждение о том, что все во
всех последующих ячейках $c[i][j]$ записан пустой список. В нашем случае, инвариант будет выглядеть следующим образом:

\begin{figure}[H]
  \caption{Инвариант внешнего цикла}
  \label{code:outer_cycle_inv}
  \begin{minted}[]{coq}
    xfor_inv (fun (i:int) => 
    Hexists (x' : list (list int)),
    p1 ~> Array l1 \*
    p2 ~> Array l2 \*
    c ~> Array x' \*
    \[length x' = (n+1)*(m+1)] \*
    \[forall i1 i2 : int, 0 <= i1 < i -> 0 <= i2 <= m -> 
        Lcs x'[i1*(m+1) + i2] (take i1 l1 ) (take i2 l2) ] \* 
    \[forall i', i*(m+1) <= i' < (n+1)*(m+1) ->
        x'[i'] = nil ]). 
  \end{minted}
\end{figure}

Два последних коснтантных предиката кучи в этом выражении как раз и постулируют описанные выше условия. Остальные требования переносят
предыдущие переменные неизменными в контекст цикла, а также требование $c \sim> \text{Array}\ x'$  как раз постулирует, что $c$ указывает на
список с нужными свойствами. Написанный выше инвариант относится к внешнему циклу, аналогичный инвариант нужно написать и для внутреннего цикла:

\begin{figure}[H]
  \caption{Инвариант внутреннего цикла}
  \label{code:inner_cycle_inv}
  \begin{minted}[]{coq}
  xfor_inv (fun (j:int) => 
    Hexists (x' : list (list int)),
    p1 ~> Array l1 \*
    p2 ~> Array l2 \*
    c ~> Array x' \*
    \[length x' = (n+1)*(m+1)] \*
    \[forall i1 i2 : int, 0 <= i1 <= i -> 0 <= i2 <= m -> 
        i1*(m+1) + i2 < i*(m+1) + j -> 
        Lcs x'[i1*(m+1) + i2] (take i1 l1 ) (take i2 l2) ] \*
    \[forall i', i*(m+1) + j <= i' < (n+1)*(m+1) ->
        x'[i'] = nil ]). 
  \end{minted}
\end{figure}

После формулировки данных инвариантов, оставшаяся часть доказательства представляет из себя довольно занудную процедуру
разбора случаев, в которых применяется $lcs\_app\_eq$ либо $lcs\_app\_neq$ в зависимости от самого случая. Также постоянно приходится
убеждаться в том, что при обращении к массиву мы не выходим за его рамки.
\subsection{Доказательство асимптотической корректности}
Как было изложено раньше, фреймворк CFML \cite{base_article} позволяет нам построить функцию стоимости алгоритма в процессе
доказательства алгоритмической корректности. После этого, нам нужно показать, что полученная функция является монотонной, а
также проверить, что она доминируется заявленной функцией, то есть в нашем случае функцией $f(n,m) = nm$. В процессе доказательства
корректности генерируется следующая функция стоимости:

\begin{figure}[H]
  \caption{Функция стоимости}
  \label{code:cost_function}
  \begin{minted}[]{coq}
    fun '(x, y) =>
         (1 +
          (1 +
           (1 +
            ((((x + 1)%I * (y + 1)%I)%I + 1)%I +
             (cumul 1 (x + 1)
                (fun _ : int =>
                 1 +
                 cumul 1 (y + 1)
                   (fun _ : int =>
                    1 +
                    (1 +
                     (1 +
                      (0 +
                       Z.max (1 + (1 + (0 + 1)%I)%I)
                         (1 + (0 + (1 + (0 + Z.max (1 + 1) (1 + 1))%I)%I)%I))%I)%I)%I)) +
              (1 + 1)%I)%I)%I)%I)%I)%I
  \end{minted}
\end{figure}

Это выражение можно сильно упростить, если заметить, что в аргументе cumul стоит константная функция. После упрощения получаем:

\begin{figure}[H]
  \caption{Упрощенная функция стоимости}
  \label{code:simplified_cost_function}
  \begin{minted}[]{coq}
  (fun '(x, y) =>
   (((((x * y)%I + x)%I + y)%I +
     (Z.max 0 ((x + 1)%I - 1) *
      (1 +
       (Z.max 0 ((y + 1)%I - 1) *
        (1 + (1 + (1 + Z.max (1 + (1 + 1)%I) 
        (1 + (1 + (1 + 1)%I)%I))%I)%I)%I)%I)%I)%I)%I + 7)%I)
  \end{minted}
\end{figure}

Из ее вида уже сразу видно, что она монотонна, и доминируется. Монотонность проверяется непосредственно тактикой \textcolor{blue}{math\_nia}.
Для проверки того, что она доминируется функцией $f(n, m) = n*m$, мы для начала раскладываем нашу функцию на слагаемые, и показываем,
что каждое из них доминируется. Разложение на слагаемые делается с помощью дистрибутивности по сложению, то есть утверждения:
$f_1 = \OO(g) \land f_2 = \OO(g) \implies f_1 + f_2 = \OO(g)$, которое формализуется следующим образом:


\begin{figure}[H]
  \caption{Дистрибутивность доминирования по сложению}
  \label{code:dominated_sum_distr_nary}
  \begin{minted}[]{coq}
    Theorem dominated_sum_distr_nary
    : forall (domain : list Type) (M : Filter.mixin_of (Rtuple domain))
          (f g h : Rarrow domain int),
        dominated (nFilterType domain M) (Uncurry f) (Uncurry h) ->
        dominated (nFilterType domain M) (Uncurry g) (Uncurry h) ->
        dominated (nFilterType domain M)
          (Fun' (fun p : Rtuple domain => (App f p + App g p)%I)) 
          (Uncurry h)
  \end{minted}
\end{figure}
После разложения на слагаемые получаются довольно простые функции, доминируемость которых мы показываем с помощью дистрибутивности
доминирования умножению, то есть утверждения о том, что \\
$f_1 = \OO(g_1) \land f_2 = \OO(g_2) \implies f_1f_2 = \OO(g_1g_2)$, которое формально выглядит так:

\begin{figure}[H]
  \caption{Дистрибутивность доминирования по умножению}
  \label{code:dominated_mul_nary}
  \begin{minted}[]{coq}
    Theorem dominated_mul_nary
	 : forall (domain : list Type) (M : Filter.mixin_of (Rtuple domain))
         (f1 f2 g1 g2 : Rarrow domain int),
       dominated (nFilterType domain M) (Uncurry f1) (Uncurry g1) ->
       dominated (nFilterType domain M) (Uncurry f2) (Uncurry g2) ->
       dominated (nFilterType domain M)
         (Fun' (fun p : Rtuple domain => (App f1 p * App f2 p)%I))
         (Fun' (fun p : Rtuple domain => (App g1 p * App g2 p)%I))
  \end{minted}
\end{figure}

С помощью этих утверждений, а также некоторых утверждений о свойствах фильтров, мы завершаем доказательство.
С полной формализацие результата можно ознакомиться в приложении, а также в репозитории \cite{Grigoryants2021}.