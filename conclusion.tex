В нашей работе мы на примере формальной верификации корректности алгоритма LCS и его асимптотики применили методы, описанные в \cite{base_article}
на практике, формально верифицировав утверждение \ref{code:main_result_definition}.
В процессе применения этой библиотеки, возникли некоторые проблемы с отсутствием квантора всеобщности для предиката кучи,
а именно, без него не получается строить утверждения о двумерных, и, следовательно, многомерных списках. Одним из вариантов развития
этой работы может послужить реализация предиката \textcolor{blue}{Hforall}, а также доказательство его различных свойств и интеграция в тактику
\textcolor{blue}{hsimpl}. Этот предикат был бы очень полезен не только при работе с многомерными массивами, но и во многих других случаях.
Еще одним интересным вариантом развития идей формальной верификации асимптотики алгоритмов является формальная верификация асимптотики
вероятностных алгоритмов. В то время как формализация проверки корректности вероятностных алгоритмов изучается довольно активно
(\cite{AUDEBAUD2009568}, \cite{tassarotti2017verifying}, \cite{10.1007/978-3-319-46750-4_5}), проверка асимптотики вероятностных алгоритмов
до сих пор практически не рассматривалась. Рассмотренные в этой работе методы могут быть обобщены и на вероятностные алгоритмы.
